%-------------------------------------------------------------------------------
%	PACKAGES EN DOCUMENT CONFIGURATIE
%-------------------------------------------------------------------------------

\documentclass[hidelinks]{uva-inf-article}
\usepackage[english]{babel}

\usepackage{tikz}
\usetikzlibrary{automata,positioning}

\usepackage{listings}

%-------------------------------------------------------------------------------
%	GEGEVENS VOOR IN DE TITEL
%-------------------------------------------------------------------------------

% Vul de naam van de opdracht in.
\assignment{Assignment 6}
% Vul het soort opdracht in.
\assignmenttype{Essay}
% Vul de titel van de eindopdracht in.
\title{Code Generation}

% Vul de volledige namen van alle auteurs in.
\author{René Kok}
\uvanetid{13671146}

\author{Aram Mutlu}
\uvanetid{13574116}

% Vul eventueel ook de naam van de docent of vakcoordinator toe.
\docent{Dhr. dr. C.U. Grelck}
\course{Compiler Construction}
% Te vinden op onder andere Datanose.
\courseid{5062COMP6Y}

% Dit is de datum die op het document komt te staan. Standaard is dat vandaag.
\date{\today}

%-------------------------------------------------------------------------------
%	VOORPAGINA
%-------------------------------------------------------------------------------

\begin{document}
\maketitle

%-------------------------------------------------------------------------------
%	INHOUDSOPGAVE EN ABSTRACT
%-------------------------------------------------------------------------------

% Niet doen bij korte verslagen en rapporten
%\tableofcontents
%\begin{abstract}
%\lipsum[13]
%\end{abstract}

%-------------------------------------------------------------------------------
%	INTRODUCTIE
%-------------------------------------------------------------------------------

\section{Introduction}
\begin{flushleft}
\par This report provides information regarding the sixth assignment 
"Code Generation" of the course Compiler Construction by Dhr.
dr. C.U. Grelck. The goal of this assignment is to manually generate CiviC-VM 
assembly code, point out the relationship between assembly code and source code,
add the number of bytes required for each line of CiviC-VM assembly code, compute 
the proper byte code offset for each jump instruction and lastly devise a 
compilation scheme that replaces each occurrence of a for-loop in the body of 
a CiviC function by semantically equivalent CiviC code
%-------------------------------------------------------------------------------
%	METHODE
%-------------------------------------------------------------------------------

\newpage
\section{Code Generation}
Consider the following CiviC function definition:
\begin{lstlisting}[basicstyle=\small, language=C, label=lst:code, caption=CiviC function definition, captionpos=b]
    int factorial ( int x )
    {
        int res ;
        if ( x <= 1) res = 1;
        else res = x * factorial ( x - 1);
        return res ;
    }
\end{lstlisting}

\textbf{A) Manually generate CiviC-VM assembly code for the above function definition. Make use of
labels to mark destinations of jump instructions.}

\begin{lstlisting}[basicstyle=\small, language=C, label=lst:code, caption=Assembly code, captionpos=b]
    factorial:    
        esr 1               // Add factorial function

        iload_0             // Add (local) int x to stack
        iloadc 0            // Add constant 1 to stack
        ile                 // Int less or equal (x <= 1)
        branch_f L1         // Continue at L1 if above value is true
        iloadc 0            // Add constant 1 to stack 
        istore 1            // Assign constant 1 to res 
        jump L2             // Jump to L2 (skip the else statement (L1:))

    L1:                     // L1 label
        iload_0             // Load (local) var x
        isrg factorial      // Load factorial() function 
        iload_0             // Load (local) var x
        iloadc 0            // Load constant 1
        isub                // Subtract the numbers x - 1
        jsr factorial 1     // Call factorial() with 1 param
        imul                // Multiply the outcome -> x*factorial(x-1)
        istore 1            // Store the outcome (res = ...outcome)

    L2:                     // end label (if statement)
        iload_1             // Add res to stack
        ireturn             // Return int res 
\end{lstlisting}



\textbf{B) Point out the relationship between assembly code and source code through line by line
comments in the assembly code.}
\par Answer added to Listing of answer A above.

\textbf{C) Add the number of bytes required for each line of CiviC-VM assembly code. Assume here
jump instructions would take byte code offsets as arguments and not labels.}
\begin{lstlisting}[basicstyle=\small, language=C, label=lst:code, caption=Assembly code with number of bytes, captionpos=b]
    factorial:    
        esr 1               // 2 bytes
    
        iload_0             // 1 byte
        iloadc 0            // 3 bytes
        ile                 // 1 byte
        branch_f L1         // 2 bytes      // Jump 7 bytes
        iloadc 0            // 3 bytes
        istore 1            // 2 bytes
        jump L2             // 2 bytes      // Jump 14 bytes
    
    L2:                   
        iload_0             // 1 byte
        isrg factorial      // 2 bytes
        iload_0             // 1 byte
        iloadc 0            // 3 bytes
        isub                // 1 byte
        jsr factorial 1     // 3 bytes      // Jump -26 bytes
        imul                // 1 bytes
        istore 1            // 2 bytes
    
    L2:                    
        iload_1             // 1 byte
        ireturn             // 1 byte
    \end{lstlisting}
    

\textbf{D) Compute the proper byte code offset for each jump instruction. Consult the CiviC-VM
manual for details on individual instructions.}
\par Answer added to Listing of answer C above.

\newpage
\section{Compilation Schemes Revisited}
\textbf{Devise a compilation scheme that replaces each occurrence of a for-loop in the body of a CiviC
function by semantically equivalent CiviC code that makes use of while-loops and/or do/whileloops instead. As a simplification consider only for-loops without a step specification, and assume
that CiviC would support arbitrary interleaving of variable declarations and statements in function
bodies as in C99 proper.}

\begin{lstlisting}[caption=Compilation Scheme, captionpos=b]

C [[ for (int i = lower, upper) { body } Rest ]]
-> CX [[
        while(lower != upper){ 
            C[[body]] lower = lower + 1 
        } 
        C[[Rest]] 
    ]]
\end{lstlisting}
\end{flushleft}
\end{document}