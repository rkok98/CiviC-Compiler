%-------------------------------------------------------------------------------
%	PACKAGES EN DOCUMENT CONFIGURATIE
%-------------------------------------------------------------------------------

\documentclass[hidelinks]{uva-inf-article}
\usepackage[english]{babel}

\usepackage{tikz}
\usetikzlibrary{automata,positioning}

\usepackage{listings}

%-------------------------------------------------------------------------------
%	GEGEVENS VOOR IN DE TITEL
%-------------------------------------------------------------------------------

% Vul de naam van de opdracht in.
\assignment{Assignment 6}
% Vul het soort opdracht in.
\assignmenttype{Essay}
% Vul de titel van de eindopdracht in.
\title{Code Generation}

% Vul de volledige namen van alle auteurs in.
\author{René Kok}
\uvanetid{13671146}

\author{Aram Mutlu}
\uvanetid{13574116}

% Vul eventueel ook de naam van de docent of vakcoordinator toe.
\docent{Dhr. dr. C.U. Grelck}
\course{Compiler Construction}
% Te vinden op onder andere Datanose.
\courseid{5062COMP6Y}

% Dit is de datum die op het document komt te staan. Standaard is dat vandaag.
\date{\today}

%-------------------------------------------------------------------------------
%	VOORPAGINA
%-------------------------------------------------------------------------------

\begin{document}
\maketitle

%-------------------------------------------------------------------------------
%	INHOUDSOPGAVE EN ABSTRACT
%-------------------------------------------------------------------------------

% Niet doen bij korte verslagen en rapporten
%\tableofcontents
%\begin{abstract}
%\lipsum[13]
%\end{abstract}

%-------------------------------------------------------------------------------
%	INTRODUCTIE
%-------------------------------------------------------------------------------

\section{Introduction}
\begin{flushleft}
\par This report provides information regarding the sixth assignment 
"Code Generation" of the course Compiler Construction by Dhr.
dr. C.U. Grelck. The goal of this assignment is to  Manually generate CiviC-VM assembly code, apply the loop unswitching optimisation and
devise a formal compilation scheme that systematically eliminates all occurrences of 
while-loops.
%-------------------------------------------------------------------------------
%	METHODE
%-------------------------------------------------------------------------------

\newpage
\section{Code Generation}
Consider the following CiviC function definition:
\begin{lstlisting}[basicstyle=\small, language=C, label=lst:code, caption=CiviC function definition, captionpos=b]
    int factorial ( int x )
    {
        int res ;
        if ( x <= 1) res = 1;
        else res = x * factorial ( x - 1);
        return res ;
    }

\end{lstlisting}

\textbf{Manually generate CiviC-VM assembly code for the above function definition. Make use of
labels to mark destinations of jump instructions.}


\textbf{Point out the relationship between assembly code and source code through line by line
comments in the assembly code.}


\textbf{Add the number of bytes required for each line of CiviC-VM assembly code. Assume here
jump instructions would take byte code offsets as arguments and not labels.}


\textbf{Compute the proper byte code offset for each jump instruction. Consult the CiviC-VM
manual for details on individual instructions.}

\end{flushleft}
\end{document}