%-------------------------------------------------------------------------------
%	PACKAGES EN DOCUMENT CONFIGURATIE
%-------------------------------------------------------------------------------

\documentclass[hidelinks]{uva-inf-article}
\usepackage[english]{babel}

\usepackage{tikz}
\usetikzlibrary{automata,positioning}

\usepackage{listings}

\lstset{literate=%
{=>}{$\Rightarrow{}$}{2}
{ε}{$\epsilon{}$}{1}
}

%-------------------------------------------------------------------------------
%	GEGEVENS VOOR IN DE TITEL
%-------------------------------------------------------------------------------

\assignment{Assignment 3}
\assignmenttype{Essay}
\title{Syntactic Analysis}

\author{René Kok}
\uvanetid{13671146}

\author{Aram Mutlu}
\uvanetid{13574116}

\docent{Dhr. dr. C.U. Grelck}
\course{Compiler Construction}
\courseid{5062COMP6Y}

\date{\today}

\begin{document}
\maketitle

%-------------------------------------------------------------------------------
%	INTRODUCTIE
%-------------------------------------------------------------------------------

\section{Introduction}
\begin{flushleft}
\par This report provides information regarding the third assignment 
     "Syntactic Analysis" of the course Compiler Construction lectured by Dhr. dr. C.U. Grelck. 
     The goal of this assignment is to derive pseudo code for a top-down recursive-descent parse. 
     To accomplish this we will start to rewrite the given grammar according to the C standard. 
     From there we can convert the left-recursive grammar into a right-recursive grammar and convert it into a start-seperated, predictive grammar.
     With this solution we can create pseudo code for the goal of this assignment.
     The given grammar for this assignment:
\begin{lstlisting}
Expr    =>  Id
        |   Expr + Expr
        |   - Expr
        |   Expr ++
        |   ( Expr )
\end{lstlisting}
%-------------------------------------------------------------------------------
%	METHODE
%-------------------------------------------------------------------------------
\newpage
\section{Precedence and Associativity}
\par The first part of the assignment is to rewrite the given grammar such that properly
expresses precedence and associativity according to the C standard. In C there are 17 precedence levels to check on.
So we have to device our grammar into multiple levels. For this grammar we need 4 levels (+, -, ++ and (...) / id). 
If we separate these we will get the following grammar:
\begin{figure}[h]
\begin{lstlisting}
Expr4   =>  Expr4 + Expr3 
        |   Expr3

Expr3   =>  - Expr3 
        |   Expr2

Expr2   =>  Expr2 ++ 
        |   Expr1
        
Expr1   =>  ( Expr4 ) 
        |   Id
\end{lstlisting}
\caption{Grammar of expressions with proper precedence and associativity}
\label{fig:1}
\end{figure}
\newpage
\section{Left- and Right-recursive Grammars}
\par Based on the grammar we created in the first part of the assignment, we can now convert it into a
right-recursive grammar. A right-recursive grammar is said to be right recursive if it has right recursion.
This means the tree of the grammar will grow to the right. If we change our grammar into a right-recursive grammar,
we will get:
\begin{figure}[h]
\begin{lstlisting}
Expr4   =>  Expr3 Expr4'
Expr4'  =>  + Expr4 Expr3
        |   ε
    
Expr3   =>  Expr2 Expr3'
Expr3'  =>  - Expr3
        |   ε
    
Expr2   =>  Expr1 Expr2'
Expr2'  =>  ++ Expr2
        |   ε
    
Expr1   =>  Id Expr1'
Expr1'  =>  ( Expr4 )
        |   ε
\end{lstlisting}
\caption{Right-recursive grammar of expressions with proper precedence and associativity}
\label{fig:2}
\end{figure}
\par In this grammar we can see that each level in the grammar gets a new inner level named with prime and ends with an $\epsilon{}$ (epsilon)
which means an end of line.

\section{Predictive Grammars}
Following grammar is a start-seperated and predictive grammar.
A predictive grammar is one where it's possible decide the right rule by looking at the first token or first N tokens.
\begin{figure}[h]
\begin{lstlisting}
Start   =>  Expr4

Expr4   =>  Expr3 Expr4'
Expr4'  =>  + Expr4 Expr3
        |   ε

Expr3   =>  Expr2 Expr3'
Expr3'  =>  - Expr3
        |   ε

Expr2   =>  Expr1 Expr2'
Expr2'  =>  ++ Expr2
        |   ε

Expr1   =>  Id Expr1'
Expr1'  =>  ( Expr4 )
        |   ε
\end{lstlisting}
\caption{Right-recursive grammar of expressions with proper precedence and associativity}
\label{fig:3}
\end{figure}
\newpage
\section{Recursive-descent Parsing}
\lstinputlisting[basicstyle=\small, language=C]{parser.c}
\end{flushleft}
\end{document}