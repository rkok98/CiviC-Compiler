%-------------------------------------------------------------------------------
%	PACKAGES EN DOCUMENT CONFIGURATIE
%-------------------------------------------------------------------------------

\documentclass[hidelinks]{uva-inf-article}
\usepackage[english]{babel}

\usepackage{tikz}
\usetikzlibrary{automata,positioning}

\usepackage{listings}

\lstset{literate=%
{=>}{$\Rightarrow{}$}{2}
{ε}{$\epsilon{}$}{1}
}

%-------------------------------------------------------------------------------
%	GEGEVENS VOOR IN DE TITEL
%-------------------------------------------------------------------------------

\assignment{Assignment 3}
\assignmenttype{Essay}
\title{Syntactic Analysis}

\author{René Kok}
\uvanetid{13671146}

\author{Aram Mutlu}
\uvanetid{13574116}

\docent{Dhr. dr. C.U. Grelck}
\course{Compiler Construction}
\courseid{5062COMP6Y}

\date{\today}

\begin{document}
\maketitle

\begin{flushleft}

%-------------------------------------------------------------------------------
%	METHODE
%-------------------------------------------------------------------------------

\section{Precedence and Associativity}
\begin{figure}[h]
\begin{lstlisting}
Expr4   =>  Expr4 + Expr3 
        |   Expr3

Expr3   =>  - Expr3 
        |   Expr2

Expr2   =>  Expr2 ++ 
        |   Expr1
        
Expr1   =>  ( Expr4 ) 
        |   Id
\end{lstlisting}
\caption{Grammar of expressions with proper precedence and associativity}
\label{fig:1}
\end{figure}
\newpage
\section{Left- and Right-recursive Grammars}
\begin{figure}[h]
\begin{lstlisting}
Expr4   =>  Expr3 Expr4'
Expr4'  =>  + Expr4 Expr3
        |   ε
    
Expr3   =>  Expr2 Expr3'
Expr3'  =>  - Expr3
        |   ε
    
Expr2   =>  Expr1 Expr2'
Expr2'  =>  ++ Expr2
        |   ε
    
Expr1   =>  Id Expr1'
Expr1'  =>  ( Expr4 )
        |   ε
\end{lstlisting}
\caption{Right-recursive grammar of expressions with proper precedence and associativity}
\label{fig:2}
\end{figure}
\section{Predictive Grammars}
Following grammar is a start-seperated and predictive grammar.
A predictive grammer is one where it's possible decide the right rule by looking at the first token or first N tokens.
\begin{figure}[h]
\begin{lstlisting}
Start   =>  Expr4

Expr4   =>  Expr3 Expr4'
Expr4'  =>  + Expr4 Expr3
        |   ε

Expr3   =>  Expr2 Expr3'
Expr3'  =>  - Expr3
        |   ε

Expr2   =>  Expr1 Expr2'
Expr2'  =>  ++ Expr2
        |   ε

Expr1   =>  Id Expr1'
Expr1'  =>  ( Expr4 )
        |   ε
\end{lstlisting}
\caption{Right-recursive grammar of expressions with proper precedence and associativity}
\label{fig:3}
\end{figure}
\newpage
\section{Recursive-descent Parsing}
\lstinputlisting[basicstyle=\small, language=C]{parser.c}
\end{flushleft}
\end{document}