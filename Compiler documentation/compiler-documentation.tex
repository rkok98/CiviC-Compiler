%%%%%%%%%%%%%%%%%%%%%%%%%%%%%%
% LATEX-TEMPLATE TECHNISCH RAPPORT
%-------------------------------------------------------------------------------
% This template is derived from the AVI PT report template.
%%%%%%%%%%%%%%%%%%%%%%%%%%%%%%

%-------------------------------------------------------------------------------
%	PACKAGES EN DOCUMENT CONFIGURATIE
%-------------------------------------------------------------------------------

\documentclass{uva-inf-article}
%\usepackage[dutch]{babel}
%\usepackage{csquotes}

\usepackage[style=numeric-comp]{biblatex}
\addbibresource{PT-essay.bib}

%-------------------------------------------------------------------------------
%	GEGEVENS VOOR IN DE TITEL
%-------------------------------------------------------------------------------

% Vul de naam van de opdracht in.
\assignment{Individual Written Assessment}
% Vul het soort opdracht in.
\assignmenttype{Essay}
% Vul de titel van de eindopdracht in.
\title{Compiler construction}

% Vul de volledige namen van alle auteurs in.
\author{Name}
% Vul de corresponderende UvAnetID's in.
\uvanetid{UvAnetID}

% Vul eventueel ook de naam van de docent of vakcoordinator toe.
\docent{}
\course{Programmeertalen}
% Te vinden op onder andere Datanose.
\courseid{}

% Dit is de datum die op het document komt te staan. Standaard is dat vandaag.
\date{\today}

%-------------------------------------------------------------------------------
%	VOORPAGINA
%-------------------------------------------------------------------------------

\begin{document}
\maketitle

%-------------------------------------------------------------------------------
%	INHOUDSOPGAVE EN ABSTRACT
%-------------------------------------------------------------------------------

% Niet doen bij korte verslagen en rapporten
%\tableofcontents
%\begin{abstract}
%\lipsum[13]
%\end{abstract}

%-------------------------------------------------------------------------------
%	INTRODUCTIE
%-------------------------------------------------------------------------------

\section{Introduction}
%\lipsum[1]

%-------------------------------------------------------------------------------
%	METHODE
%-------------------------------------------------------------------------------

\section{Analysis}
%\lipsum[3]

\section{Discussion}
%\lipsum[5]

\section{Conclusions}
%\lipsum[7]

%-------------------------------------------------------------------------------
%	REFERENTIES
%-------------------------------------------------------------------------------

%\printbibliography

%-------------------------------------------------------------------------------
%	BIJLAGEN EN EINDE
%-------------------------------------------------------------------------------

%\section{Bijlage A}
%\section{Bijlage B}
%\section{Bijlage C}
\end{document}
